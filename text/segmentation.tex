\documentclass{scrartcl}

\usepackage{parskip}
\usepackage{a4wide}
\usepackage{amsmath,amsfonts}

\usepackage{multicol}
\setlength{\columnsep}{1cm}

\title{Enhancing scalability of Peer-to-Peer energy markets using adaptive segmentation method}
\author{}
\date{}

\begin{document}

\maketitle

%\begin{multicols*}{2}

\ldots

\section*{Market clearing method}

\ldots	

\subsection*{Segmentation}

Market segmentation is one of the most fundamental
strategic marketing concepts, which can be used to group
players according to their similarity in several dimensions
related to a product under consideration. Segmenting a
market means dividing its potential consumers into separate
sub-sets where players in the same group are similar with
respect to a given set of characteristics. Cluster analysis
allows reducing the number of observations, by grouping
them into homogeneous clusters.

In the P2P energy trading, as there are a large number of
players, market segmentation can be used to divide market
to several segment, where in each segment only a few
players with similar features negotiate for energy trading.
In this paper, an adaptive segmentation method is proposed
to divide a large scale market into several subgroups, where
two important characteristics are considered to form
segments; capacity (to secure trading amount), and price (to
improve/minimize utility/cost). 
For each player $i$ a bid
vector $\omega_i$ is submitted to the market.
\begin{equation}
	\omega_i = \{ \bar{X_i}, \gamma_i \}
\end{equation}
where, $\bar{X_i}$ and $\gamma_i$ are maximum power and its
corresponding price for player $i$ which indicates a point
of marginal cost/benefit curve of player. A set of $N_s$
segments is generated before segmentation and players will
be clustered in these typical segments separately. Historical
data can be used to set initial segments. Determining the
number of segments without prior information is a non-trivial and computationally expensive problem. The
segmentation method used in this paper is distance-based,
where market players are assigned to the different segments
$j \in \{1, \ldots, N_s\}$ . It should be noted that in each segment the total
demand and supply should be balanced and this constraint
need to be taken in account during the segmentation.

After assigning players to different segments, at the first
step centroid of each segment is calculated by averaging parameters of all members in each segment. The centroid
of segment jj is presented by (2)

Then, distance of all players with centroids is calculated
and stored in a distance matrix. Distance parameter is
obtained as

where, ζζ xx , ζζ λλ are scaling factors. The distance matrix can
be obtained by
An allocation vector specifying to which typical segment
each player is assigned to is generated using

These steps will repeat until there is no more new
assignment.

%\end{multicols*}
\end{document}
\documentclass{scrartcl}

\usepackage{parskip}
\usepackage{a4wide}
\usepackage{amsmath,amsfonts}
\usepackage{todonotes}

\usepackage{multicol}
\setlength{\columnsep}{1cm}

\title{Enhancing scalability of Peer-to-Peer energy markets using adaptive segmentation method}
\author{}
\date{}

\begin{document}

\maketitle

\begin{multicols*}{2}

\ldots

\section*{Market clearing method}

\ldots	

\subsection*{Segmentation}

Market segmentation is one of the most fundamental
strategic marketing concepts, which can be used to group
players according to their similarity in several dimensions
related to a product under consideration. Segmenting a
market means dividing its potential consumers into separate
sub-sets where players in the same group are similar with
respect to a given set of characteristics. Cluster analysis
allows reducing the number of observations, by grouping
them into homogeneous clusters.

In the P2P energy trading, as there are a large number of
players, market segmentation can be used to divide market
to several segment, where in each segment only a few
players with similar features negotiate for energy trading.
In this paper, an adaptive segmentation method is proposed
to divide a large scale market into several subgroups, where
two important characteristics are considered to form
segments; capacity (to secure trading amount), and price (to
improve/minimize utility/cost). 
For each player $i$ a bid
vector $\omega_i$ is submitted to the market.
\begin{equation}
	\omega_i = \{ \bar{X_i}, \gamma_i \}
\end{equation}
where, $\bar{X_i}$ and $\gamma_i$ are maximum power and its
corresponding price for player $i$ which indicates a point
of marginal cost/benefit curve of player. A set of $N_s$
segments is generated before segmentation and players will
be clustered in these typical segments separately. Historical
data can be used to set initial segments. Determining the
number of segments without prior information is a non-trivial and computationally expensive problem. The
segmentation method used in this paper is distance-based,
where market players are assigned to the different segments
$j \in \{1, \ldots, N_s\}$. This problem can be formulated as follows:

\begin{align}
\min_{\mathbf{y}} & \sum_{i=1}^{n} \sum_{j=1}^{k} y_{ij} \big[ (\bar{X_i} - x_j)^2 + (\gamma_i - \lambda_j)^2 \big] \nonumber \\
\text{s.t.} & \nonumber \\
& \sum_{j=1}^k y_{ij} = 1 \qquad \forall i \\
& x_j = \frac{\sum_{i=1}^{n} y_{ij} \bar{X_i}}{\sum_{i=1}^{n} y_{ij}}  \qquad \forall j \label{cons:centroid1}\\
& \lambda_j = \frac{\sum_{i=1}^{n} y_{ij} \gamma_i}{\sum_{i=1}^{n} y_{ij}} \qquad \forall j \label{cons:centroid2}\\
& x_{ij} \in \{0, 1\} \qquad \forall i, j \nonumber
\end{align}

It is known that removing constraints~\ref{cons:centroid1} and~\ref{cons:centroid2} does not change the optimal solution of this problem. In other words, for each possible grouping of points to clusters, if we can freely choose the cluster centers to minimize the total distance of points to these centers, the optimal centers will be located at the centroids of clusters~\cite{AloiseHL12}.\todo[inline]{rework the previous sentence}  

Solving this nonlinear problem to optimality is computationally intensive~\cite{AloiseDHP09}. The poplular \emph{KMeans} algorithm provides an approximate solution to this problem through a sequence of iterations, as follows: 

\begin{enumerate}
\item Assign values to $x_j$ and $\lambda_j$ variables. 
\item Fix the values of $x_j$ and $\lambda_j$ variables and solve  for $y_{ij}$ variables.
\item Fix the $y_{ij}$ variables and solve for $x_j$ and $\lambda_j$ variables.
\end{enumerate}

The iteration between steps 2 and 3 of this algorithm continues until convergence, that is, until the clustering obtained in step 2 does not change in two consecutive iterations. A notable advantage of KMeans algorithm is that the subproblems of steps 2 and 3 have closed-form solutions. Given a fixed set of centers, the optimal assignment is obtained by assigning each point to the closest center. As stated before, the optimal centers for each given clustering are located at the centroids of the clusters. 

It should be noted that in each segment the total
demand and supply should be balanced and this constraint
need to be taken in account during the segmentation. Let us denote the supply of player $i$ by $x_i$. If player $i$ is a buyer, $x_i$ is a negative number and represents its demand. Our goal is to solve the clustering problem subject to these extra constraints:

\begin{equation}
	\underline{\theta} \leq \sum_{i=1}^{n} y_{ij} x_i \leq \bar{\theta} \qquad \forall j 
\end{equation}

where $\underline{\theta}$ and $\bar{theta}$ represent the bounds on degree of imbalance in each cluster. This problem can be solved using the same approach as KMeans, except that the subproblem of step 2 changes into the following problem:

\begin{align*}
\min_{\mathbf{y}} & \sum_{i=1}^{n} \sum_{j=1}^{k} y_{ij} \big[ (\bar{X_i} - \hat{x_j})^2 + (\gamma_i - \hat{\lambda_j})^2 \big] \\
\text{s.t.} & \\
& \sum_{j=1}^k y_{ij} = 1 \qquad \forall i \\
& \underline{\theta} \leq \sum_{i=1}^{n} y_{ij} x_i \leq \bar{\theta} \qquad \forall j \\
& x_{ij} \in \{0, 1\} \qquad \forall i, j
\end{align*}

where constants $\hat{x_j}$ and $\hat{\lambda_j}$ represent the centroids computed in previous steps of the algorithm.

This problem is an integer linear program which can be solved using available solvers. However, the cost of solving an integer program at each iteration of the algorithm can be prohibitive. An alternative method is to use a local search method for solving this problem.  
 
\bibliographystyle{plain}
\bibliography{references}

\end{multicols*}
\end{document}